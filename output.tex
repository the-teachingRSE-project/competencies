% Options for packages loaded elsewhere
\PassOptionsToPackage{unicode}{hyperref}
\PassOptionsToPackage{hyphens}{url}
%
\documentclass[a4paper
]{article}
\usepackage{lmodern}
\usepackage{amssymb,amsmath}
\usepackage{ifxetex,ifluatex}
\ifnum 0\ifxetex 1\fi\ifluatex 1\fi=0 % if pdftex
  \usepackage[T1]{fontenc}
  \usepackage[utf8]{inputenc}
  \usepackage{textcomp} % provide euro and other symbols
\else % if luatex or xetex
  \usepackage{unicode-math}
  \defaultfontfeatures{Scale=MatchLowercase}
  \defaultfontfeatures[\rmfamily]{Ligatures=TeX,Scale=1}
\fi
% Use upquote if available, for straight quotes in verbatim environments
\IfFileExists{upquote.sty}{\usepackage{upquote}}{}
\IfFileExists{microtype.sty}{% use microtype if available
  \usepackage[]{microtype}
  \UseMicrotypeSet[protrusion]{basicmath} % disable protrusion for tt fonts
}{}
\makeatletter
\@ifundefined{KOMAClassName}{% if non-KOMA class
  \IfFileExists{parskip.sty}{%
    \usepackage{parskip}
  }{% else
    \setlength{\parindent}{0pt}
    \setlength{\parskip}{6pt plus 2pt minus 1pt}}
}{% if KOMA class
  \KOMAoptions{parskip=half}}
\makeatother
\usepackage{xcolor}
\IfFileExists{xurl.sty}{\usepackage{xurl}}{} % add URL line breaks if available
\IfFileExists{bookmark.sty}{\usepackage{bookmark}}{\usepackage{hyperref}}
\hypersetup{
  pdftitle={Teaching and Learning Research Software Engineering},
  pdfauthor={Heidi Seibold; Florian Goth; Jan Linxweiler; Jan Philipp Thiele; Jeremy Cohen; Renato Alves; Samatha Wittke; Jean-Noël Grad; Fredo Erxleben},
  hidelinks,
  pdfcreator={LaTeX via pandoc}}
\urlstyle{same} % disable monospaced font for URLs
\usepackage[margin=2.5cm]{geometry}
\usepackage{longtable,booktabs}
% Correct order of tables after \paragraph or \subparagraph
\usepackage{etoolbox}
\makeatletter
\patchcmd\longtable{\par}{\if@noskipsec\mbox{}\fi\par}{}{}
\makeatother
% Allow footnotes in longtable head/foot
\IfFileExists{footnotehyper.sty}{\usepackage{footnotehyper}}{\usepackage{footnote}}
\makesavenoteenv{longtable}
\setlength{\emergencystretch}{3em} % prevent overfull lines
\providecommand{\tightlist}{%
  \setlength{\itemsep}{0pt}\setlength{\parskip}{0pt}}
\setcounter{secnumdepth}{-\maxdimen} % remove section numbering
\usepackage{pdflscape}
\newcommand{\blandscape}{\begin{landscape}}
\newcommand{\elandscape}{\end{landscape}}
\usepackage[]{biblatex}
\addbibresource{bibliography.bib}

\title{Teaching and Learning Research Software Engineering}
\author{Heidi Seibold \and Florian Goth \and Jan Linxweiler \and Jan
Philipp Thiele \and Jeremy Cohen \and Renato Alves \and Samatha
Wittke \and Jean-Noël Grad \and Fredo Erxleben}
\date{August 4, 2023}

\begin{document}
\maketitle

{
\setcounter{tocdepth}{3}
\tableofcontents
}
\hypertarget{working-title-identifying-core-competencies-of-an-rse-and-an-application-with-a-sample-curriculum.}{%
\subsection{Working Title: Identifying core competencies of an RSE and
an application with a sample
curriculum.}\label{working-title-identifying-core-competencies-of-an-rse-and-an-application-with-a-sample-curriculum.}}

\begin{center}\rule{0.5\linewidth}{0.5pt}\end{center}

\textbf{Abstract}: Being an outcome of a community workshop held in
Paderborn, Germany in February 2023 this paper tries for the first
time(FIXME: ? true?) to define which competencies are required to
participate in modern digital sciences. Some of these competencies are
required in more depth, therefore, giving rise to the trade of the RSE,
scientific personnel that specializes in writing research software that
facilitates research in all stages of the research cycle. Due to their
generality, we explore these competencies in various contexts and
elaborate on some examples for further specialization.

But knowing a set of competencies is not enough, therefore we discuss
explicitly how to make people aware that these skills are required and
how these are taught(FIXME: Do we want to add this pedagogical
dimension?). In order to also facilitate structural change in the German
research institution landscape we will discuss the organizations and
structures that support this change and educate new RSEs. The discussion
in this paper is meant to be general therefore we will discuss domain
specific applications in an appendix.

\begin{center}\rule{0.5\linewidth}{0.5pt}\end{center}

\textbf{Keywords}: research software engineering, training, learning,
competencies

\hypertarget{introduction}{%
\subsection{Introduction}\label{introduction}}

\begin{itemize}
\tightlist
\item
  background

  \begin{itemize}
  \tightlist
  \item
    RSEs have been around for a long time but the name is new.
  \item
    RSEs have a specialist skill set that brings together technical and
    research knowledge.
  \item
    Skills development traditionally provided largely through peer
    learning, self learning, introductory training courses not targeted
    specifically at RSEs, \ldots{}
  \item
    Requirements for RSE skills growing rapidly across all domains.
  \end{itemize}
\item
  past attempts, other initiatives
\item
  contributions
\end{itemize}

Computers and software have played a key role in the research lifecycle
for many decades. Traditionally, they were specialist tools used only in
a small number of fields and the Computer Scientists who maintained and
programmed them needed extensive technical training over several years
to gain the necessary skills and expertise. Fast forward 50-60 years and
software and computation are all around us, underpinning our everyday
lives. This shift is also true within research.

With the ability to capture and process ever more data, undertake larger
scale, higher resolution simulations and, increasingly, automate complex
tasks through Artificial Intelligence and Machine Learning approaches,
computers and software are now vital elements of the research process
across almost all domains. However, this shift means that there is a
much greater need for core research software skills across a vast array
of research fields where these were not previously required.

The people who focus on writing research software are now known as
Research Software Engineers (RSEs) - a term that was coined a little
over 10 years ago \autocite{Hettrick2016}. RSEs may work within one of
the many Research Software Engineering teams that have been set up at
universities and research organisations over the last decade, or they
may be embedded within a research team. They may have a job title that
officially recognises them as an RSE, or they may have a standard
research or technical job title such as Research Assistant, Research
Fellow or Software Engineer. Regardless of their job title, RSEs share a
set of core skills that are required to write software, understand the
research environment and ensure that they produce sustainable,
maintainable code that supports reproducible research outputs. In order
to do so the draw upon skills from traditional software engineering,
established research culture and a commitment to being part of a team.

Developing and maintaining these skills is time consuming and often
challenging. Part of the challenge is that there is not a standard
pathway to becoming an RSE and, partly as a result of this, there is
something of an ad hoc approach to training within the community. We
also see increasing amounts of basic-level training materials that are
great to put researchers or technical professionals on a path towards
gaining significant RSE expertise, but the trail often ends as
developing RSEs want to progress to intermediate and advanced level
material. In particular, recent technology developments are ensuring
that there is a growing need for specialist expertise, for example in
areas such as making efficient use of high-performance computing
infrastructure. This is an area where there is a skills shortage and a
shortage of training materials.

In this paper, we look at the challenge of understanding the core
competencies that underpin Research Software Engineering and the way
that these competencies may be combined to help support a more
coordinated approach to future RSE skills development. The paper builds
on a workshop session held as part of the German Research Software
Engineering Conference (deRSE23), held in Paderborn, Germany in February
2023.

\emph{{[}Information on key contributions to add{]}}

\emph{{[}Overview of paper sections to add{]}}

\hypertarget{related-work-and-activities}{%
\subsection{Related Work and
Activities}\label{related-work-and-activities}}

The challenges of understanding the current state of skills within the
research software community and related areas, as well as identifying
required competencies, developing training pathways and providing
training materials are areas that are being looked at and addressed by
various groups and projects. In this section, we highlight some of these
other projects and activities.

\hypertarget{identifying-skills-and-pathways}{%
\subsubsection{Identifying skills and
pathways}\label{identifying-skills-and-pathways}}

As an area that generally requires a range of advanced skills, High
Performance Computing (HPC) is one field where there is ongoing work to
identify relevant sets of skills for HPC practitioners and potential
paths to develop these skills. The HPC Certification Forum has developed
a ``competence standard'' (CS) for HPC that defines a range of skills
and how they are related in the context of a skill tree
\autocite{HPCCFCompetencies}, \autocite{Kunkel2020a},
\autocite{Kunkel2020b}. This competence standard is currently being
built upon by the CASTIEL 2 \autocite{CASTIEL2} project in collaboration
with initiatives funded by the European High Performance Computing Joint
Undertaking (EuroHPC JU) to create a framework for HPC certification
\autocite{EuroHPCJU2023}. Also looking at pathways and how different
skills are related, the UNIVERSE-HPC project \autocite{UNIVERSEHPC},
funded under the UK's ExCALIBUR research programme \autocite{EXCALIBUR},
is looking to understand and develop training pathways to support the
development of specialist skills in the HPC and exascale domains. The
project is gathering open source training materials to develop curricula
that support the training pathways that are underpinned by high-quality
training materials.

\begin{itemize}
\tightlist
\item
  There are some projects / papers looking at skills pathways - if we're
  going to include a separate section on related work, as proposed here,
  this should probably be expanded to include more of this content?
\item
  FIXME: include the ELIXIR part here
\end{itemize}

\hypertarget{rse-related-training-materials}{%
\subsubsection{RSE-related Training
Materials}\label{rse-related-training-materials}}

A wide range of software-related training materials and supporting
organisations exist within the research software community and beyond.

\hypertarget{the-carpentries}{%
\paragraph{The Carpentries}\label{the-carpentries}}

The Carpentries \autocite{Carpentries} is a non-profit entity that
supports a range of open source training materials and international
communities of volunteer instructors and helpers who run courses around
these materials. The community also maintains the materials which are
based around three core syllabuses - Software Carpentry, Data Carpentry
and Library Carpentry. The training materials within these areas have
been developed, reviewed and enhanced over several years ensuring that
they represent best practice in training on these topics. The core
Carpentries lessons are targeted primarily at the beginner level.
However, the Carpentries Incubator \autocite{CarpentriesIncubator}
provides an environment for hosting additional community-developed
training modules covering a wide range of other topics that have not
gone through the peer review process of the core lessons. The material
in the Incubator increasingly includes more intermediate-level training
modules.

\hypertarget{coderefinery}{%
\paragraph{Coderefinery}\label{coderefinery}}

\autocite{CodeRefinery} is a project currently funded by the Nordic
e-Infrastructure and thus active primarily in the Nordics with the goal
of teaching essential tools around research software development, that
are usually skipped in academic education. CodeRefinery hosts a set of
open source training materials including both beginner and intermediate
level material and organizes multiple highly interactive large scale
workshops per year. Skills learned from the workshops and/or materials
allow researchers to produce more reproducible, open and efficient
software and thus promote FAIR research practices. One goal of the
project is to evolve into a community project that seamlessly integrates
with other initiatives. FIXME: elaborate on the integration part if it's
relevant, else leave out.

\hypertarget{reprohack}{%
\paragraph{Reprohack}\label{reprohack}}

The ReproHack Team offers resources to host events where students and
researchers can get together to try and reproduce the results of
published papers with the methods described there or ideally with the
software provided by the authors. FIXME: This is applied FAIR principles
elaborate a bit on why this is related for RSE competencies? Or do we
want to make the point, that this teaches reproducible science?

\hypertarget{prace}{%
\paragraph{PRACE}\label{prace}}

The Partnership for Advanced Computing in Europe (PRACE)
\autocite{PRACE} offers training in the form of massive open online
courses (MOOCs), online and on-site training events at European HPC
facilities (aggregated on various websites, e.g.~EuroCC Training
\autocite{EuroCCTraining}), and white papers. While most training events
are tailored for HPC-RSE, several recurring courses about programming
languages (C++, Fortran, Python) are suitable for general RSEs, as they
teach coding best practices, modern software design
\autocite{LRZModernCpp}, project management and version control
\autocite{LRZIntroCpp}.

\hypertarget{helmholtz}{%
\paragraph{Helmholtz}\label{helmholtz}}

As part of its push towards a better RSE environment, the Helmholtz
Association launched the \autocite{HIFIS} platform which provides
educational material and trainings amongst other services for an
audience of over 10.000 scientists in Germany and internationally. All
of these materials focus on RSE basics to refresh and expand the
software engineering knowledge for recent graduates or to update the
existing knowledge in established researchers. They are published under
OER licenses and can serve as either self-learning instructions or form
the basis of a hands-on training. To allow these educational offers to
be easier brought to the scientists, the \autocite{HIDA} sustains a
large network within the Helmholtz Association and beyond with a strong
focus on graduate schools. Further RSE training offers within the
Helmholtz context are provided by the \autocite{HAI} and
\autocite{HImaging} platforms as well as the \autocite{HMC}.

\hypertarget{open-source-resources}{%
\paragraph{Open Source Resources}\label{open-source-resources}}

Due to the ever-evolving nature of skills and infrastructure in the RSE
field, training material is often version-controlled, so that trainers
can update it between iterations. For example, core lessons from the
Carpentries and CodeRefinery are stored on GitHub, and any change is
automatically mirrored to their website. Likewise, the reference work on
RSE by Fogel \autocite{Fogel2005} was released in its second edition as
a living document \autocite{Fogel2017}.

Reference works are also available for self-study \autocite{Fogel2005},
\autocite{Irving2021}.

The National Competence Center Sweden (ENCCS) {[}https://enccs.se/{]}
provides instructor training material
\autocite{ENCCSInstructorTraining}, \autocite{ENCCS2022} developed from
Carpentries and CodeRefinery material, as well as lessons for
HPC-oriented RSEs \autocite{ENCCSLessons}.

The (Intersect){[}https://intersect-training.org/{]} project \ldots.

(BSSW){[}https://bssw.io/{]} \ldots{}

SSI? \autocite{Crouch2013} \ldots{}

\begin{itemize}
\tightlist
\item
  FIXME: Add NFDI initiatives like edutrain.
\end{itemize}

\hypertarget{challenges}{%
\subsection{Challenges}\label{challenges}}

\begin{itemize}
\tightlist
\item
  Point out gaps
\item
  What is missing
\item
  domain application?
\end{itemize}

Depending on the specific domain there is a gap between the basic
software carpentry courses and the required skills to build
domain-specific research software. For example, scientists in the field
of High Performance Computing (HPC) need to know how to make effective
use of concurrency to speed up their simulations and communicate
efficiently using message-passing interface (MPI) libraries. The same is
true for researchers from other domains who make use of other
specialized technologies, methods and/or tools. To bridge those gaps
more specialized courses would be needed like the one mentioned in
section \protect\hyperlink{identifying-skills-and-pathways}{Identifying
skills and pathways} for the HPC community.

Moreover, software development is a craft, i.e.~it is not only about
knowledge but also requires practical experience. Hence we need to
create an environment that allows less experienced researchers to
practice and gain experience efficiently. Ideally, this learning
environment would allow less experienced scientists to be guided by more
experienced RSEs. We know such practices e.g.~from human medicine, where
junior doctors first assist experienced doctors before they work
independently. In the field of software development, this approach could
be implemented in the form of peer programming, for example. The
prerequisite for this, however, is that experienced academics get better
career opportunities at German universities so that they do not leave
for industry roles.

\hypertarget{results}{%
\subsection{Results}\label{results}}

\hypertarget{required-generic-rse-skills}{%
\subsubsection{Required Generic RSE
skills}\label{required-generic-rse-skills}}

As it stands the RSE role requires competencies in two fields. The
``R'', the person being a researcher, and the ``SE'' the software
skills. And this hybrid nature is brought about, since RSEs need to
apply their knowledge usually in teams. Therefore we structure our
competencies among SE skills, research skills and team skills with key
notions being the software and the research cycle and the scientific
process. Since these skills are meant to be relevant in a broad setting
and form the foundation for a specific specialization. We elaborate on
some facets in tables.

\hypertarget{software-engineering-skills}{%
\paragraph{Software Engineering
Skills}\label{software-engineering-skills}}

There are lots of software engineering curricula out there, that try to
define which tasks a software engineer should be able to perform. A
recent one highlighting some aspects in more detail than what we are
doing here is \autocite{Landwehr2017}.

\hypertarget{creating-documented-code-building-blocks-docbb}{%
\paragraph{Creating documented code building blocks
(DOCBB)}\label{creating-documented-code-building-blocks-docbb}}

The RSE should be able to create building blocks from source code that
are reusable. This ranges from simple libraries of functions up to
complex architectures consisting of multiple softwares. An important
part of reusability is that at least oneself, and ideally others, are
able to understand what a piece of code aims to do and how to use the
provided functionality, which is primarily achieved through a ``clean''
implementation and enhanced by documentation. This ranges from
commenting code blocks to the usage of documentation (building) tools.

\hypertarget{building-distributable-libraries-libs}{%
\paragraph{Building distributable libraries
(LIBS)}\label{building-distributable-libraries-libs}}

The RSE should be able to distribute their code with their
domain/language specific distribution platforms. This almost always
encompasses handling/documenting dependencies to other
packages/libraries and sometimes requires knowledge of using build
systems to enable interoperability with other systems.

\hypertarget{understanding-the-software-lifecycle-swlc}{%
\paragraph{Understanding the software lifecycle
(SWLC)}\label{understanding-the-software-lifecycle-swlc}}

Software has a lifetime and this necessitates the respective strategies
for its usage along the intended time scale.

\hypertarget{use-repositories-swrepos}{%
\paragraph{Use repositories (SWREPOS)}\label{use-repositories-swrepos}}

The RSE should be able to use public platforms to share the artifacts
they have created and invite public scrutiny on them for public review.

\hypertarget{legal-things-leg}{%
\paragraph{Legal things (LEG)}\label{legal-things-leg}}

The RSE should know licenses and their respective domains for data or
software. On an entry level, the competency is mostly about awareness.
Namely that different (open source) licenses exist, that those might not
be combinable when using multiple libraries with different licenses and
that use of third party software might restrict licensing of the
resulting work.

\hypertarget{software-behaviour-awareness-and-analysismod}{%
\paragraph{Software Behaviour Awareness and
Analysis(MOD)}\label{software-behaviour-awareness-and-analysismod}}

By this, we mean a certain quality of analytical thinking that enables
you to form a mental model of the piece of software under consideration
in the current environment. Using that, an RSE should be able to make
predictions about a software's behaviour. This is a required skill for
tasks like debugging, profiling, designing good tests, or predicting
user interaction.

\hypertarget{the-research-skills}{%
\paragraph{The research skills}\label{the-research-skills}}

\hypertarget{curiosity-new}{%
\paragraph{Curiosity (NEW)}\label{curiosity-new}}

The RSE gains its reputation from its effectiveness to interact with
their domain peers. Therefore some curiosity together with a broad
overview of the research field is required. A manifestation can also be
the curiosity for new tools which is a great asset for an RSE. Lifelong
learning then becomes more bearable.

\hypertarget{understanding-the-research-cycle-rc}{%
\paragraph{Understanding the research cycle
(RC)}\label{understanding-the-research-cycle-rc}}

Knowing that ones own research is not only a means to personal ends, but
that one is part of a bigger cycle that involves a lot of other parties
in and outside of your domain should foster an appreciation for the
underlying principles of science like review and reproducibility.

\hypertarget{findingdiscovering-software-and-attribution-sd}{%
\paragraph{Finding/discovering software and attribution
(SD)}\label{findingdiscovering-software-and-attribution-sd}}

One goal of FAIR software is to avoid reimplementation of already
working packages and thereby reducing the need for doubled work. To
(re-) use software the individual researchers have to be able to find
out if that software already exists.

After finding the software the researcher has to be able to evaluate if
the software actually suits their needs. Apart from the functionality,
licensing, integration with other software and expandability have to be
part of this evaluation.

Finally, after obtaining results by modifying and/or using the software,
it has to be ensured that the original authors get the proper
attribution.

\hypertarget{use-domain-repositoriesdirectories-domrep}{%
\paragraph{Use Domain repositories/directories
(DOMREP)}\label{use-domain-repositoriesdirectories-domrep}}

Almost all research software is developed within a specific scientific
domain. Some software may be able to cross boundaries, but the majority
will have a home domain, with which it needs to be able to interact.
Especially for data-driven research having software that is able to use
existing sets and repositories is a valuable part. The RSE should be
able to interact with the repositories of this specific domain.

\hypertarget{outside-party-interaction-users}{%
\paragraph{Outside Party interaction
(USERS)}\label{outside-party-interaction-users}}

While in a traditional SE context, you might get away with not
interacting with people outside your project. But in a research context,
this will certainly be the case and involves users, other developers,
upto funders. Additionally, this is oftentimes a two-way interaction
with RSEs in a specific domain learning new findings, techniques,
algorithms, etc. to be able to implement software that is up-to-date
with the body of knowledge of that domain.

\hypertarget{team-skills}{%
\paragraph{Team Skills}\label{team-skills}}

\hypertarget{teaching-teach}{%
\paragraph{Teaching (TEACH)}\label{teaching-teach}}

Working in a group means being able to effectively perform
e.g.~onboarding, or more formal teaching procedures to their colleagues.
This includes tasks such as consulting and mentoring since these also
often aim at educating people. We deliberately mention, that giving code
reviews is also part of this skill, since Code review can be part of
teaching people on improving their skills.

\hypertarget{project-management-pm}{%
\paragraph{Project Management (PM)}\label{project-management-pm}}

The RSE should have knowledge about project management. At some
institutes, it follows the practices of the local research groups, but
it is useful, if an RSE knows its place in a PM scheme, or can bring in
new ideas for improvement.

\hypertarget{working-in-a-team-team}{%
\paragraph{working in a team (TEAM)}\label{working-in-a-team-team}}

There are various facets to working in a team. They range from
functioning in a team to leading a team. It includes following measures
that increase team cohesion like performing code reviews.

\hypertarget{current-day-contextualization}{%
\paragraph{Current Day
Contextualization}\label{current-day-contextualization}}

These skills, while already numerous are also on purpose generic.
Concrete examples can be obtained by the outcome from the Paderborn
workshop, where we asked learners and beginner RSEs of what they would
liked to have learnt. Among the top five things mentioned were:

\begin{itemize}
\tightlist
\item
  Testing. This task is a manifestation of the SE competencies of DOCBB
  and MOD since a model of the software is required in order to write
  good test that facilitate understanding and documentation. Today this
  encompasses the knowledge of testing frameworks and CI/CD practices.
\item
  Contributing to large projects. This is a topic that requires
  competency in SWREPOS , LEG, in order to understand the ramifications
  of sharing, DOCBB, since the contributed code has to be understood by
  others, and TEAM, since one is becoming part of the project depending
  on the involvement. Today this entails the effecive use of
  collaborative platforms like github/gitlab, honoring a projects Code
  of Conduct, and some knowledge of software licenses like the GPL.
\item
  When or why to keep repositories private. This decision requires
  knowledge in the RC, to understand when it makes sense, USERS and TEAM
  in order to do accepted decisions and sometimes LEG. This requires
  domain and location knowledge in the sense that one should know what
  the practices of ones own institution are.
\item
  Proper Development. This broad topic requires all of the SE skills. Of
  course these are the competencies that are the most fluid since they
  have to adapt at a high rate to the technological advancements.
  Additionally proper SE skills often require knowledge of TEAM, and PM.
  Today this means effective use of IDEs, static analysis tools, design
  patterns, documentation (for oneself and others), etc.
\item
  Finding a community. This can be interpreted in two different facets.
  First we have the aspect of community building for a research project.
  Since this deals with software that is supposed to be used in research
  this requires knowledge of RC, USERS, and also NEW, in order to
  effectively interact with domain scientists. Today, an example is a
  presence on social media. The other TEAM-related aspect is the
  embedding of RSE graduates into the community of RSEs. We envision our
  RSE graduates to be a part in a strong network of other RSEs,
  tool-related communities and the classical domain communities. This
  point is further elaborated in
  \protect\hyperlink{how-do-we-reach-people-in-different-stages-of-their-careers}{How
  do we reach people in different stages of their careers}
\end{itemize}

\hypertarget{how-much-do-different-people-need-to-know}{%
\subsubsection{How much do different people need to
know?}\label{how-much-do-different-people-need-to-know}}

Now that we have the different competencies, we can explore various
dimensions of these competencies, depending on their circumstances. A
strong beneficiary of specialized RSEs can also be newly formed RSE
centers at research institutions.

\hypertarget{career-level}{%
\paragraph{Career level}\label{career-level}}

At different career levels, differing skills are required. We have set
this up according to the following separation often applied within a
single project:

\begin{itemize}
\tightlist
\item
  Junior RSE: These are persons that have just started, but generally
  speaking they should have the skills to contribute to software
  projects
\item
  Senior RSE: They have gained experience and can set the examples in
  the software project.
\item
  Principal RSE: Their actual job description varies a lot. These may be
  RSE team leaders based in a professional services type role, or they
  may be professors or research group leaders based in a more
  academic-focused role. They are often the people responsible for
  bringing in the money that supports new projects and sustains existing
  projects. Generally speaking, they do not need to to be actively
  involved in the day-to-day technical tasks but they should be able to
  guide projects from both a technical and research perspective.
\end{itemize}

The required skills are distributed according to this table First
Dimension: Career path e.g.~Junior RSE -\textgreater{} Senior RSE
-\textgreater{} PI scale (1-\textgreater6) (less -\textgreater{} lot)

\begin{landscape}

\begin{longtable}[]{@{}llll@{}}
\toprule
\begin{minipage}[b]{0.19\columnwidth}\raggedright
\strut
\end{minipage} & \begin{minipage}[b]{0.28\columnwidth}\raggedright
Junior\strut
\end{minipage} & \begin{minipage}[b]{0.28\columnwidth}\raggedright
Senior\strut
\end{minipage} & \begin{minipage}[b]{0.14\columnwidth}\raggedright
Principal RSE(brings in funding)\strut
\end{minipage}\tabularnewline
\midrule
\endhead
\begin{minipage}[t]{0.19\columnwidth}\raggedright
DOCBB\strut
\end{minipage} & \begin{minipage}[t]{0.28\columnwidth}\raggedright
should be able to write reusable building blocks\strut
\end{minipage} & \begin{minipage}[t]{0.28\columnwidth}\raggedright
same as junior, but the quality should set the standard for the project,
while following current best practices\strut
\end{minipage} & \begin{minipage}[t]{0.14\columnwidth}\raggedright
should know the current best practices and point its people to the right
resources.\strut
\end{minipage}\tabularnewline
\begin{minipage}[t]{0.19\columnwidth}\raggedright
LIBS\strut
\end{minipage} & \begin{minipage}[t]{0.28\columnwidth}\raggedright
should be able to use package distribution platforms\strut
\end{minipage} & \begin{minipage}[t]{0.28\columnwidth}\raggedright
same as junior, but should set the project standard and follow current
best practices.\strut
\end{minipage} & \begin{minipage}[t]{0.14\columnwidth}\raggedright
should ensure that their project is in an up-to-date distribution
platform\strut
\end{minipage}\tabularnewline
\begin{minipage}[t]{0.19\columnwidth}\raggedright
MOD\strut
\end{minipage} & \begin{minipage}[t]{0.28\columnwidth}\raggedright
should have a basic grasp of their piece of the software in order to use
basic tools like a debugger\strut
\end{minipage} & \begin{minipage}[t]{0.28\columnwidth}\raggedright
Should understand the characteristics of large parts of the codebase
considering a variety of the metrics\strut
\end{minipage} & \begin{minipage}[t]{0.14\columnwidth}\raggedright
Should understand the big idea of the software project in order to
define the task that the software solves\strut
\end{minipage}\tabularnewline
\begin{minipage}[t]{0.19\columnwidth}\raggedright
SWLC\strut
\end{minipage} & \begin{minipage}[t]{0.28\columnwidth}\raggedright
Awareness about the existence of the software lifecycle.\strut
\end{minipage} & \begin{minipage}[t]{0.28\columnwidth}\raggedright
Should know which decisions lead to technical debt.\strut
\end{minipage} & \begin{minipage}[t]{0.14\columnwidth}\raggedright
Should know in which part of the lifecycle their project is and how to
steer development/project resources accordingly.\strut
\end{minipage}\tabularnewline
\begin{minipage}[t]{0.19\columnwidth}\raggedright
SWREPOS\strut
\end{minipage} & \begin{minipage}[t]{0.28\columnwidth}\raggedright
Seamless interaction with the swrepo of their project is a must\strut
\end{minipage} & \begin{minipage}[t]{0.28\columnwidth}\raggedright
Should be well-versed in the intricacies of a swrepo, and probably
interact with multiple projects' repo's\strut
\end{minipage} & \begin{minipage}[t]{0.14\columnwidth}\raggedright
Should be able to effectively interact with swrepos and especially the
interaction with the connecting projects.\strut
\end{minipage}\tabularnewline
\begin{minipage}[t]{0.19\columnwidth}\raggedright
LEG\strut
\end{minipage} & \begin{minipage}[t]{0.28\columnwidth}\raggedright
Awareness about legal intricacies about sharing code\strut
\end{minipage} & \begin{minipage}[t]{0.28\columnwidth}\raggedright
Should be able to give advice on legal issues and resolve the most
common issues\strut
\end{minipage} & \begin{minipage}[t]{0.14\columnwidth}\raggedright
same as Senior RSE\strut
\end{minipage}\tabularnewline
\begin{minipage}[t]{0.19\columnwidth}\raggedright
NEW\strut
\end{minipage} & \begin{minipage}[t]{0.28\columnwidth}\raggedright
Some curiosity required to fit into research teams\strut
\end{minipage} & \begin{minipage}[t]{0.28\columnwidth}\raggedright
same as junior, but a curiosity to enhance the code base is
required\strut
\end{minipage} & \begin{minipage}[t]{0.14\columnwidth}\raggedright
Curiosity to know in which direction to steer the project is
required\strut
\end{minipage}\tabularnewline
\begin{minipage}[t]{0.19\columnwidth}\raggedright
RC\strut
\end{minipage} & \begin{minipage}[t]{0.28\columnwidth}\raggedright
Awareness about the RC\strut
\end{minipage} & \begin{minipage}[t]{0.28\columnwidth}\raggedright
should know the position of the project in the RC\strut
\end{minipage} & \begin{minipage}[t]{0.14\columnwidth}\raggedright
Should know what is necessary for the project to fit into its position
in the RC\strut
\end{minipage}\tabularnewline
\begin{minipage}[t]{0.19\columnwidth}\raggedright
SD\strut
\end{minipage} & \begin{minipage}[t]{0.28\columnwidth}\raggedright
Should be aware about tools for SD\strut
\end{minipage} & \begin{minipage}[t]{0.28\columnwidth}\raggedright
Should be able to find sth. with SD tools\strut
\end{minipage} & \begin{minipage}[t]{0.14\columnwidth}\raggedright
Should be able to effectively find sth. with SD tools and be able to
evaluate and perform the integration of a library into the
project.\strut
\end{minipage}\tabularnewline
\begin{minipage}[t]{0.19\columnwidth}\raggedright
DOMREP\strut
\end{minipage} & \begin{minipage}[t]{0.28\columnwidth}\raggedright
The RSE should be able to interact with the domain repository\strut
\end{minipage} & \begin{minipage}[t]{0.28\columnwidth}\raggedright
same as junior RSE\strut
\end{minipage} & \begin{minipage}[t]{0.14\columnwidth}\raggedright
same as junior, and should know about how it fits into workflows
surrounding these domain repositories\strut
\end{minipage}\tabularnewline
\begin{minipage}[t]{0.19\columnwidth}\raggedright
USERS\strut
\end{minipage} & \begin{minipage}[t]{0.28\columnwidth}\raggedright
The RSE should be able to communicate with non-SE users of the
project\strut
\end{minipage} & \begin{minipage}[t]{0.28\columnwidth}\raggedright
same as junior\strut
\end{minipage} & \begin{minipage}[t]{0.14\columnwidth}\raggedright
same as junior, and take feedback into account of the steering\strut
\end{minipage}\tabularnewline
\begin{minipage}[t]{0.19\columnwidth}\raggedright
TEACH\strut
\end{minipage} & \begin{minipage}[t]{0.28\columnwidth}\raggedright
should be able to perform simple peer-to-peer onboarding tasks\strut
\end{minipage} & \begin{minipage}[t]{0.28\columnwidth}\raggedright
should be able to explain logical components to other RSEs\strut
\end{minipage} & \begin{minipage}[t]{0.14\columnwidth}\raggedright
Should be able to effectively communicate about all large-scale parts of
the project.\strut
\end{minipage}\tabularnewline
\begin{minipage}[t]{0.19\columnwidth}\raggedright
PM\strut
\end{minipage} & \begin{minipage}[t]{0.28\columnwidth}\raggedright
Awareness about the employed project managemement method\strut
\end{minipage} & \begin{minipage}[t]{0.28\columnwidth}\raggedright
Should be able to use the employed PM method\strut
\end{minipage} & \begin{minipage}[t]{0.14\columnwidth}\raggedright
Should be able to design and adapt the employed PM method.\strut
\end{minipage}\tabularnewline
\begin{minipage}[t]{0.19\columnwidth}\raggedright
TEAM\strut
\end{minipage} & \begin{minipage}[t]{0.28\columnwidth}\raggedright
Should be able to work in the team in order to effectively fulfill the
given tasks. Should be able to learn from code review.\strut
\end{minipage} & \begin{minipage}[t]{0.28\columnwidth}\raggedright
Should be able to break down tasks into more easily digestable
sub-tasks\strut
\end{minipage} & \begin{minipage}[t]{0.14\columnwidth}\raggedright
Should be able to lead the team and set the respective direction.\strut
\end{minipage}\tabularnewline
\bottomrule
\end{longtable}

\end{landscape}

\hypertarget{academic-progression-career-path-help-me-for-better-title}{%
\paragraph{Academic Progression / Career Path? (Help me for better
title)}\label{academic-progression-career-path-help-me-for-better-title}}

Modern digital science requires some digital proficiency at every level.
To be a bit more precise, these are how we define the academic levels:

\begin{itemize}
\tightlist
\item
  Bachelor: These are people in their undergrad studies, that mostly
  consume science/knowledge. They should also learn about the existence
  of certain digital structures.
\item
  Master: Ultimately, their study should have brought them to a level,
  where they can participate in science, hence they should be able to
  use ``some'' digital structures.
\item
  PhD: Under guidance they perform independent research and hence they
  should get to know all relevant structures.
\item
  PostDoc: Independent researchers, they are proficient users of all
  tools.
\item
  PI/Professor: Experts in their field, they should be able to give
  proper guidance to their students on which digital tools are currently
  relevant.
\end{itemize}

It is important to note that the following table does not reflect the
current state of academic training and research institutions. Instead,
it summarizes the discussions with and between workshop participants at
different levels of academic progression on what they would have liked
to learn at an earlier stage or know before starting their current
position. While individuals already work at implementing some of these
changes and teaching these skills it has not yet reached a systemic
level.

Additionally, this table tries to cover all domains that rely on
software tools in at least a basic level. Certain fields, e.g.~sciences
relying on simulations, might require higher skill levels in the SE
competencies as software development is a large part of their actual
research.

\begin{landscape}

\begin{longtable}[]{@{}llllll@{}}
\toprule
\begin{minipage}[b]{0.12\columnwidth}\raggedright
\strut
\end{minipage} & \begin{minipage}[b]{0.18\columnwidth}\raggedright
Bachelor\strut
\end{minipage} & \begin{minipage}[b]{0.18\columnwidth}\raggedright
Master\strut
\end{minipage} & \begin{minipage}[b]{0.09\columnwidth}\raggedright
PhD\strut
\end{minipage} & \begin{minipage}[b]{0.18\columnwidth}\raggedright
PostDoc\strut
\end{minipage} & \begin{minipage}[b]{0.09\columnwidth}\raggedright
PI/Professor\strut
\end{minipage}\tabularnewline
\midrule
\endhead
\begin{minipage}[t]{0.12\columnwidth}\raggedright
DOCBB\strut
\end{minipage} & \begin{minipage}[t]{0.18\columnwidth}\raggedright
They should be aware that RSEs exist and that software has different
quality aspects\strut
\end{minipage} & \begin{minipage}[t]{0.18\columnwidth}\raggedright
Same as Bachelor\strut
\end{minipage} & \begin{minipage}[t]{0.09\columnwidth}\raggedright
They should know where they can get help, and maybe able to use
libraries\strut
\end{minipage} & \begin{minipage}[t]{0.18\columnwidth}\raggedright
same as PhD\strut
\end{minipage} & \begin{minipage}[t]{0.09\columnwidth}\raggedright
They should know the skills of an RSE and when they might need one in
their group\strut
\end{minipage}\tabularnewline
\begin{minipage}[t]{0.12\columnwidth}\raggedright
LIBS\strut
\end{minipage} & \begin{minipage}[t]{0.18\columnwidth}\raggedright
They should be aware that RSEs exist and that there are tools available
in their domain\strut
\end{minipage} & \begin{minipage}[t]{0.18\columnwidth}\raggedright
They should be aware that there are tools that they can use in their
research and maybe are able to use these libraries\strut
\end{minipage} & \begin{minipage}[t]{0.09\columnwidth}\raggedright
same as Master, but able to use libraries\strut
\end{minipage} & \begin{minipage}[t]{0.18\columnwidth}\raggedright
same as PhD\strut
\end{minipage} & \begin{minipage}[t]{0.09\columnwidth}\raggedright
They should be aware of the output of RSEs and motivate their students
to use developed tools\strut
\end{minipage}\tabularnewline
\begin{minipage}[t]{0.12\columnwidth}\raggedright
MOD\strut
\end{minipage} & \begin{minipage}[t]{0.18\columnwidth}\raggedright
It is sufficient to consider digital tools as black boxes\strut
\end{minipage} & \begin{minipage}[t]{0.18\columnwidth}\raggedright
It is sufficient to be able to \emph{use} software as black boxes\strut
\end{minipage} & \begin{minipage}[t]{0.09\columnwidth}\raggedright
same as Master, but being able to write bug reports\strut
\end{minipage} & \begin{minipage}[t]{0.18\columnwidth}\raggedright
same as PhD\strut
\end{minipage} & \begin{minipage}[t]{0.09\columnwidth}\raggedright
same as PostDoc\strut
\end{minipage}\tabularnewline
\begin{minipage}[t]{0.12\columnwidth}\raggedright
SWLC\strut
\end{minipage} & \begin{minipage}[t]{0.18\columnwidth}\raggedright
Awareness of the SWLC\strut
\end{minipage} & \begin{minipage}[t]{0.18\columnwidth}\raggedright
Know that one depends on software in their own research\strut
\end{minipage} & \begin{minipage}[t]{0.09\columnwidth}\raggedright
Being able to evaluate software for their research\strut
\end{minipage} & \begin{minipage}[t]{0.18\columnwidth}\raggedright
same as PhD\strut
\end{minipage} & \begin{minipage}[t]{0.09\columnwidth}\raggedright
Should be able to judge the sustainability of the performed
research\strut
\end{minipage}\tabularnewline
\begin{minipage}[t]{0.12\columnwidth}\raggedright
SWREPOS\strut
\end{minipage} & \begin{minipage}[t]{0.18\columnwidth}\raggedright
Should know that swrepos exist\strut
\end{minipage} & \begin{minipage}[t]{0.18\columnwidth}\raggedright
same as Bachelor\strut
\end{minipage} & \begin{minipage}[t]{0.09\columnwidth}\raggedright
same as Master, but should be able to find information on them\strut
\end{minipage} & \begin{minipage}[t]{0.18\columnwidth}\raggedright
same as PhD\strut
\end{minipage} & \begin{minipage}[t]{0.09\columnwidth}\raggedright
same as PostDoc, but should be able to follow the interactions among
different projects relevant for their research\strut
\end{minipage}\tabularnewline
\begin{minipage}[t]{0.12\columnwidth}\raggedright
LEG\strut
\end{minipage} & \begin{minipage}[t]{0.18\columnwidth}\raggedright
Should know that mixing/using software has legal issues and whom to
ask\strut
\end{minipage} & \begin{minipage}[t]{0.18\columnwidth}\raggedright
same as Bachelor\strut
\end{minipage} & \begin{minipage}[t]{0.09\columnwidth}\raggedright
same as Bachelor\strut
\end{minipage} & \begin{minipage}[t]{0.18\columnwidth}\raggedright
same as PhD, but should know some simple Open Source guidelines\strut
\end{minipage} & \begin{minipage}[t]{0.09\columnwidth}\raggedright
same as PostDoc, but should know the relevant patterns for their domain
and sensitive their students\strut
\end{minipage}\tabularnewline
\begin{minipage}[t]{0.12\columnwidth}\raggedright
NEW\strut
\end{minipage} & \begin{minipage}[t]{0.18\columnwidth}\raggedright
Still consumers of lectures\strut
\end{minipage} & \begin{minipage}[t]{0.18\columnwidth}\raggedright
same as Bachelor\strut
\end{minipage} & \begin{minipage}[t]{0.09\columnwidth}\raggedright
Curiosity for their research is required, curiosity for digital tools
helpful\strut
\end{minipage} & \begin{minipage}[t]{0.18\columnwidth}\raggedright
same as PhD\strut
\end{minipage} & \begin{minipage}[t]{0.09\columnwidth}\raggedright
same as PostDoc and expert in their field\strut
\end{minipage}\tabularnewline
\begin{minipage}[t]{0.12\columnwidth}\raggedright
RC\strut
\end{minipage} & \begin{minipage}[t]{0.18\columnwidth}\raggedright
An awareness that research follows a cycle\strut
\end{minipage} & \begin{minipage}[t]{0.18\columnwidth}\raggedright
Know that research follows a cycle and locate their masters thesis'
stages in it.\strut
\end{minipage} & \begin{minipage}[t]{0.09\columnwidth}\raggedright
Same as Master, but applied to the PhD. Additionally awareness about
interaction with services\strut
\end{minipage} & \begin{minipage}[t]{0.18\columnwidth}\raggedright
Same as PhD. But proficient in the domain\strut
\end{minipage} & \begin{minipage}[t]{0.09\columnwidth}\raggedright
Same as PostDoc, but ability to lead a topic\strut
\end{minipage}\tabularnewline
\begin{minipage}[t]{0.12\columnwidth}\raggedright
SD\strut
\end{minipage} & \begin{minipage}[t]{0.18\columnwidth}\raggedright
They should know that their domain has relevant tools\strut
\end{minipage} & \begin{minipage}[t]{0.18\columnwidth}\raggedright
same as Bachelor\strut
\end{minipage} & \begin{minipage}[t]{0.09\columnwidth}\raggedright
Should know how to find full applications for their research\strut
\end{minipage} & \begin{minipage}[t]{0.18\columnwidth}\raggedright
same as PhD\strut
\end{minipage} & \begin{minipage}[t]{0.09\columnwidth}\raggedright
Should motivate their students to reuse existing tools\strut
\end{minipage}\tabularnewline
\begin{minipage}[t]{0.12\columnwidth}\raggedright
DOMREP\strut
\end{minipage} & \begin{minipage}[t]{0.18\columnwidth}\raggedright
Should be aware that their domain has repos\strut
\end{minipage} & \begin{minipage}[t]{0.18\columnwidth}\raggedright
same as Bachelor\strut
\end{minipage} & \begin{minipage}[t]{0.09\columnwidth}\raggedright
Should be able to interact with their domain repos\strut
\end{minipage} & \begin{minipage}[t]{0.18\columnwidth}\raggedright
Proficient users of their domain repos\strut
\end{minipage} & \begin{minipage}[t]{0.09\columnwidth}\raggedright
same as PostDoc\strut
\end{minipage}\tabularnewline
\begin{minipage}[t]{0.12\columnwidth}\raggedright
USERS\strut
\end{minipage} & \begin{minipage}[t]{0.18\columnwidth}\raggedright
Should be aware that they are users of a software\strut
\end{minipage} & \begin{minipage}[t]{0.18\columnwidth}\raggedright
same as Bachelor\strut
\end{minipage} & \begin{minipage}[t]{0.09\columnwidth}\raggedright
Should be aware that their user view is different from the developer, in
order to write bug reports\strut
\end{minipage} & \begin{minipage}[t]{0.18\columnwidth}\raggedright
same as PhD\strut
\end{minipage} & \begin{minipage}[t]{0.09\columnwidth}\raggedright
Should be able to contribute meaningfully to the steering decisions of
the software in their fields\strut
\end{minipage}\tabularnewline
\begin{minipage}[t]{0.12\columnwidth}\raggedright
TEACH\strut
\end{minipage} & \begin{minipage}[t]{0.18\columnwidth}\raggedright
Ability to peer-to-peer teaching\strut
\end{minipage} & \begin{minipage}[t]{0.18\columnwidth}\raggedright
Small exercise groups\strut
\end{minipage} & \begin{minipage}[t]{0.09\columnwidth}\raggedright
Ability to supervise a student.\strut
\end{minipage} & \begin{minipage}[t]{0.18\columnwidth}\raggedright
Ability to supervise students and create a course?\strut
\end{minipage} & \begin{minipage}[t]{0.09\columnwidth}\raggedright
Ability to guide students. Give full-size lectures\strut
\end{minipage}\tabularnewline
\begin{minipage}[t]{0.12\columnwidth}\raggedright
PM\strut
\end{minipage} & \begin{minipage}[t]{0.18\columnwidth}\raggedright
Awareness about project management optional\strut
\end{minipage} & \begin{minipage}[t]{0.18\columnwidth}\raggedright
Awareness that research teams are structured according to some project
management\strut
\end{minipage} & \begin{minipage}[t]{0.09\columnwidth}\raggedright
same as Master, or more depending on structure of research\strut
\end{minipage} & \begin{minipage}[t]{0.18\columnwidth}\raggedright
same as PhD\strut
\end{minipage} & \begin{minipage}[t]{0.09\columnwidth}\raggedright
Should know about the required project management they require for their
group\strut
\end{minipage}\tabularnewline
\begin{minipage}[t]{0.12\columnwidth}\raggedright
TEAM\strut
\end{minipage} & \begin{minipage}[t]{0.18\columnwidth}\raggedright
Awareness that research is often performed in groups\strut
\end{minipage} & \begin{minipage}[t]{0.18\columnwidth}\raggedright
Ability to work in their group for doing their master's thesis\strut
\end{minipage} & \begin{minipage}[t]{0.09\columnwidth}\raggedright
same as master\strut
\end{minipage} & \begin{minipage}[t]{0.18\columnwidth}\raggedright
same as master\strut
\end{minipage} & \begin{minipage}[t]{0.09\columnwidth}\raggedright
Should be able to lead a research team\strut
\end{minipage}\tabularnewline
\bottomrule
\end{longtable}

\end{landscape}

\hypertarget{project-team-size}{%
\paragraph{Project team Size}\label{project-team-size}}

Some explanation of the team sizes:

\begin{itemize}
\tightlist
\item
  individual: A single person working on their research software
\item
  Small team(\textasciitilde4 persons) This is a small team, that has
  decided to work together on something
\item
  Organizations( \textgreater10 persons): These are big organizations
  with clear structures and a bigger degree of specialization.
\end{itemize}

\begin{landscape}

\begin{longtable}[]{@{}llll@{}}
\toprule
\begin{minipage}[b]{0.19\columnwidth}\raggedright
\strut
\end{minipage} & \begin{minipage}[b]{0.28\columnwidth}\raggedright
individual\strut
\end{minipage} & \begin{minipage}[b]{0.28\columnwidth}\raggedright
small team\strut
\end{minipage} & \begin{minipage}[b]{0.14\columnwidth}\raggedright
organization\strut
\end{minipage}\tabularnewline
\midrule
\endhead
\begin{minipage}[t]{0.19\columnwidth}\raggedright
DOCBB\strut
\end{minipage} & \begin{minipage}[t]{0.28\columnwidth}\raggedright
you might get away with less satisfactory code, as long as the product
is OK\strut
\end{minipage} & \begin{minipage}[t]{0.28\columnwidth}\raggedright
think about your colleagues\strut
\end{minipage} & \begin{minipage}[t]{0.14\columnwidth}\raggedright
your organization most likely has guides here\strut
\end{minipage}\tabularnewline
\begin{minipage}[t]{0.19\columnwidth}\raggedright
LIBS\strut
\end{minipage} & \begin{minipage}[t]{0.28\columnwidth}\raggedright
you will only be successful if your artifact is usable by others\strut
\end{minipage} & \begin{minipage}[t]{0.28\columnwidth}\raggedright
same here\strut
\end{minipage} & \begin{minipage}[t]{0.14\columnwidth}\raggedright
your organization probably has rules here\strut
\end{minipage}\tabularnewline
\begin{minipage}[t]{0.19\columnwidth}\raggedright
MOD\strut
\end{minipage} & \begin{minipage}[t]{0.28\columnwidth}\raggedright
you should precisely know what your entire code is doing where\strut
\end{minipage} & \begin{minipage}[t]{0.28\columnwidth}\raggedright
you should know what your part is doing and have a feeling about the
others contributions\strut
\end{minipage} & \begin{minipage}[t]{0.14\columnwidth}\raggedright
You should know what your small part is doing\strut
\end{minipage}\tabularnewline
\begin{minipage}[t]{0.19\columnwidth}\raggedright
SWLC\strut
\end{minipage} & \begin{minipage}[t]{0.28\columnwidth}\raggedright
it's you and your software\strut
\end{minipage} & \begin{minipage}[t]{0.28\columnwidth}\raggedright
You should know the Bus factor\strut
\end{minipage} & \begin{minipage}[t]{0.14\columnwidth}\raggedright
The organization takes care of that\strut
\end{minipage}\tabularnewline
\begin{minipage}[t]{0.19\columnwidth}\raggedright
SWREPOS\strut
\end{minipage} & \begin{minipage}[t]{0.28\columnwidth}\raggedright
you need academic credit.\strut
\end{minipage} & \begin{minipage}[t]{0.28\columnwidth}\raggedright
same here\strut
\end{minipage} & \begin{minipage}[t]{0.14\columnwidth}\raggedright
your organization probably has rules here\strut
\end{minipage}\tabularnewline
\begin{minipage}[t]{0.19\columnwidth}\raggedright
LEG\strut
\end{minipage} & \begin{minipage}[t]{0.28\columnwidth}\raggedright
you carry the responsibility\strut
\end{minipage} & \begin{minipage}[t]{0.28\columnwidth}\raggedright
someone in your group needs to take car of this\strut
\end{minipage} & \begin{minipage}[t]{0.14\columnwidth}\raggedright
your organization will have specialized people for it\strut
\end{minipage}\tabularnewline
\begin{minipage}[t]{0.19\columnwidth}\raggedright
NEW\strut
\end{minipage} & \begin{minipage}[t]{0.28\columnwidth}\raggedright
You need a motivation to do this alone\strut
\end{minipage} & \begin{minipage}[t]{0.28\columnwidth}\raggedright
?\strut
\end{minipage} & \begin{minipage}[t]{0.14\columnwidth}\raggedright
Not so much, since other people might do this task\strut
\end{minipage}\tabularnewline
\begin{minipage}[t]{0.19\columnwidth}\raggedright
RC\strut
\end{minipage} & \begin{minipage}[t]{0.28\columnwidth}\raggedright
?\strut
\end{minipage} & \begin{minipage}[t]{0.28\columnwidth}\raggedright
you should maybe talk among your peers where your software fits in\strut
\end{minipage} & \begin{minipage}[t]{0.14\columnwidth}\raggedright
You define your research cycle\strut
\end{minipage}\tabularnewline
\begin{minipage}[t]{0.19\columnwidth}\raggedright
SD\strut
\end{minipage} & \begin{minipage}[t]{0.28\columnwidth}\raggedright
you need to be able to build on other work to be successful\strut
\end{minipage} & \begin{minipage}[t]{0.28\columnwidth}\raggedright
same here\strut
\end{minipage} & \begin{minipage}[t]{0.14\columnwidth}\raggedright
there might be someone in your organization who does this\strut
\end{minipage}\tabularnewline
\begin{minipage}[t]{0.19\columnwidth}\raggedright
DOMREP\strut
\end{minipage} & \begin{minipage}[t]{0.28\columnwidth}\raggedright
You're doing science in a domain\strut
\end{minipage} & \begin{minipage}[t]{0.28\columnwidth}\raggedright
there should be a person in your team who knows how to do it\strut
\end{minipage} & \begin{minipage}[t]{0.14\columnwidth}\raggedright
your organization might have specialists for that, but some basic
familiarity\strut
\end{minipage}\tabularnewline
\begin{minipage}[t]{0.19\columnwidth}\raggedright
USERS\strut
\end{minipage} & \begin{minipage}[t]{0.28\columnwidth}\raggedright
at one point you hope to have users\strut
\end{minipage} & \begin{minipage}[t]{0.28\columnwidth}\raggedright
same here\strut
\end{minipage} & \begin{minipage}[t]{0.14\columnwidth}\raggedright
maybe you have specialists for outreach\strut
\end{minipage}\tabularnewline
\begin{minipage}[t]{0.19\columnwidth}\raggedright
TEACH\strut
\end{minipage} & \begin{minipage}[t]{0.28\columnwidth}\raggedright
N/A\strut
\end{minipage} & \begin{minipage}[t]{0.28\columnwidth}\raggedright
able to peer teach\strut
\end{minipage} & \begin{minipage}[t]{0.14\columnwidth}\raggedright
teaching to groups\strut
\end{minipage}\tabularnewline
\begin{minipage}[t]{0.19\columnwidth}\raggedright
PM\strut
\end{minipage} & \begin{minipage}[t]{0.28\columnwidth}\raggedright
Not much required\strut
\end{minipage} & \begin{minipage}[t]{0.28\columnwidth}\raggedright
able to follow checklist\strut
\end{minipage} & \begin{minipage}[t]{0.14\columnwidth}\raggedright
Working with PM tools, or use them for organization\strut
\end{minipage}\tabularnewline
\begin{minipage}[t]{0.19\columnwidth}\raggedright
TEAM\strut
\end{minipage} & \begin{minipage}[t]{0.28\columnwidth}\raggedright
N/A\strut
\end{minipage} & \begin{minipage}[t]{0.28\columnwidth}\raggedright
should be able to give equal feedback to their colleagues\strut
\end{minipage} & \begin{minipage}[t]{0.14\columnwidth}\raggedright
should be able to work within their role\strut
\end{minipage}\tabularnewline
\bottomrule
\end{longtable}

\end{landscape}

Bonus points may be distributed if managing teams remotely

\href{https://competency.ebi.ac.uk/framework/bioexcel/3.0/carreer-profiles}{BIO
Excel framework}

\hypertarget{rse-specializations}{%
\subsubsection{RSE specializations}\label{rse-specializations}}

What we have defined above are intended to be base skills that an RSE
irrespective of domain, place, and time should know about. But not all
RSEs are created equal, they specialize in different areas, some of
which we want to present below. Many of the specializations may overlap,
so the same RSE might for example work on data management and on Open
Science.

\hypertarget{hpc-rse}{%
\paragraph{HPC-RSE}\label{hpc-rse}}

RSEs with a focus on High Performance Computing (HPC) have specialist
knowledge about programming models that can be used to efficiently
undertake large-scale computations on parallel computing clusters. They
may have knowledge of (automatic) code optimization tools and methods
and will understand how to write code that is optimized for different
types of computing platforms, leveraging various efficiency related
features of the target hardware. They are familiar with HPC-specific
package managers and can build dependencies from sources. They also
understand the process of interacting with job scheduling systems that
are often used on HPC clusters to manage the queuing and running of
computational tasks. HPC-focused RSEs may be involved with managing HPC
infrastructure at the hardware or software level (or both) and
understand how to calculate the environmental impact of large-scale
computations. Their knowledge of how to run HPC jobs can be vitally
important to researchers wanting to make use of HPC infrastructure.

\hypertarget{research-infrastructure-rse}{%
\paragraph{Research Infrastructure
RSE}\label{research-infrastructure-rse}}

This RSE is interested in SysOps and sets up infrastructures for and
with researchers. This RSE, therefore, requires a deep knowledge of
physical computer and network hardware. FIXME: While required, is this
an RSE? this sets off the usual infrastructure vs.~research
discussion\ldots.

\hypertarget{web-development-rse}{%
\paragraph{Web-Development RSE}\label{web-development-rse}}

This RSE is skilled in web applications, front- and/or backend, and/or
building and using APIs, for example for research data portals or big
research projects. Ideally, this RSE should also have knowledge about
(web) accessibility to allow a broad range of researchers or even the
public to use the resulting applications. Therefore a deep knowledge of
web skills is a required skill for this RSE.

\hypertarget{legal-rse}{%
\paragraph{Legal-RSE}\label{legal-rse}}

With the prevalence of software, we foresee the need for RSEs that
specialize in legal questions around software. They are the go-to person
if people have a question about licensing, mixing and matching software,
and/or patenting.

\hypertarget{data-focused-rse}{%
\paragraph{Data-focused RSE}\label{data-focused-rse}}

RSEs working at the flourishing intersection between data science and
RSE. They are skilled in cleaning data and/or running data analyses and
can help researchers in setting up their analysis pipeline and/or
research data management (RDM) solutions. When the field requires
research on sensitive data or information, e.g.~patient information in
medicine, this RSE should have knowledge about secure transfer methods
and/or ways to anonymize the data.

\hypertarget{openscience-rse}{%
\paragraph{OpenScience RSE}\label{openscience-rse}}

Open Science and FAIRness of Data and Software are increasingly
important topics in research, as exemplified by the demand of an
increasing amount of research funding agencies requiring openness. Open
Science RSEs can help researchers navigate the technical questions that
come up when practicing Open Science, such as ``How do I make my code
presentable?'', ``What do I need to consider when it comes to
licensing?'', or ``How can I use version control / automation for my
project?''.

\hypertarget{projectcommunity-manager-rses}{%
\paragraph{Project/Community manager
RSEs}\label{projectcommunity-manager-rses}}

When research software projects become larger, they need someone who
manages processes and people. Building a community around a research
project is an important building block in building sustainable software,
so these RSEs play an important role, even if they do not necessarily
touch much of the code themselves.

\hypertarget{teaching-rses}{%
\paragraph{Teaching RSEs}\label{teaching-rses}}

RSEs who focus on teaching the next generation of researchers and/or
RSEs play a vital role in quality research software.

\hypertarget{domain-rse}{%
\paragraph{\$\{DOMAIN\}-RSE}\label{domain-rse}}

While software is the lingua franca of all RSEs there will be RSEs that
have specialized in the initricacies of one particular research domain,
such as medical RSEs, digital humanities RSEs or physics RSEs.

\hypertarget{optional-rse-competencies---maintenance-rses}{%
\subsubsection{Optional RSE competencies -\textgreater{} Maintenance
RSEs}\label{optional-rse-competencies---maintenance-rses}}

Oftentimes, a significant amount of effort in (research) software
development needs to be spent on maintenance to ensure that software
remains useful for researchers now and in the future. The research
environment is constantly changing and this can also apply to the
software requirements. Accordingly, software often needs to be adapted
continuously. If it isn't, the software can reach a point where it
simply isn't useful to the researchers anymore. To avoid this, regular
work needs to be invested. While ensuring maintenance and sustainability
of research software is of huge importance to the communities that build
and use it, a particular challenge is that it's often very difficult to
obtain ongoing research funding for software maintenance tasks. As a
result, when a project that developed or extended a piece of software
finishes, it can often be the case that support for the software fades
as team members move on to other research, academic or RSE roles, or
become busy with other funded work. While this is not a core concern of
this paper, we wanted to highlight this important issue that is
frequently faced when working with software in the research community.
With regard to which additional competency is required, these are people
having experience with ancient software stacks that are not anymore part
of the general curricula(think of COBOL and FORTRAN).

FIXME: I think it would be nice if we could move each of these optional
competencies to a different specialization.

\hypertarget{how-do-we-reach-people-in-different-stages-of-their-careers}{%
\subsection{How do we reach people in different stages of their
careers?}\label{how-do-we-reach-people-in-different-stages-of-their-careers}}

Many current RSEs have found their way to being an RSE during their
doctoral studies. This transition usually happens slowly. You start
programming a tool, and someone else likes it, it becomes known that you
have programming skills and suddenly you are the RSE of the group that
everyone would like to have in their projects. If you enjoy this role,
you need to be aware that there is a RSE career path as well as that
specialized training materials exist. One place to generate awareness of
the career option and training is universities' doctoral onboarding
processes or right thereafter. RSE training could be offered as elective
courses at universities organized by some central organization. RSE
could be presented as a career path in suitable events. Since many
RSE-minded people also at some point find their way to an HPC cluster,
mailing lists of said clusters could be utilized to advertise RSE
courses. One important aspect to think about is also the wording in the
advertisement. Potential future RSEs might not know the term yet or know
that the course advertised includes topics that are of interest to them.
If the university or organization has a GitHub/Lab organization/project,
having a banner there might reach the right people. Most important is
that people working in IT helpdesks know about the courses offered so
that they can advise students/researchers on visiting the
course/reviewing the materials if related questions are asked. For an
RSE it is important to be a part of a network with other RSEs,
irrespective of the career level. A perfect first step for forming this
network is topic-related conferences, workshops or meetups. Beyond that,
there is the broader RSE community organized at the local and regional
level with chapters or local/regional communities, at the national level
with societies and the international RSE society. Each of them offers
possibilities for connecting within or beyond an individual institution
and is a great way to find like-minded people to grow a wider network
and thereby facilitate the sharing of information on interesting events
or help each other out. This available layered network can greatly
benefit the RSE in finding help with issues outside of their own comfort
zone and provides a welcoming, social safety net providing a home for
the RSE. Since we feel providing aspiring RSEs this net is of utmost
importance we envision compulsory events introducing that to young RSEs.
Qualification badges are another venue, that RSEs to find people with
the same technical interest.

Short primers on RSE skills, infrastructure and good coding practices
can be found in field-specific scientific articles and conference
proceedings, such as \autocite{Roberts1969}, \autocite{Baxter2006},
\autocite{Prlic2012}, \autocite{Leprevost2014}, \autocite{Wilson2014},
\autocite{Stodden2014}, \autocite{Crusoe2016}, \autocite{Crick2017},
\autocite{Fehr2021}, \autocite{Grossfield2022}, some of which are
specifically tailored to group leaders, institutions and scientific
journals rather than RSEs \autocite{ChueHong2013},
\autocite{ChueHong2014}, \autocite{Katerbow2018}. Scientific journals
have the advantage of reaching a large spectrum of research scientists
at all stages of their career.

\hypertarget{organizational-infrastructures}{%
\subsection{Organizational
Infrastructures}\label{organizational-infrastructures}}

So we have defined our set of competencies that we feel every RSE should
possess. Table 2 above nevertheless already hints at the fact that some
RSE skills are required during the domain studies, while Table 1 tells
us that we also need an ongoing qualification programme for people that
want to become specialized RSEs. In order to set up a proper educational
scheme we need to discuss two more items:

\begin{itemize}
\tightlist
\item
  Who are our teachers?
\item
  How is this teaching organised?
\end{itemize}

\hypertarget{the-teachers}{%
\subsubsection{The teachers}\label{the-teachers}}

\hypertarget{what-issues-are-trainers-facing-today}{%
\paragraph{What issues are trainers facing
today?}\label{what-issues-are-trainers-facing-today}}

There are already some people out there who are teaching RSE related
topics sometimes in university structures, but often outside of formal
structures. The community discussion shed some light on the issues our
trainers are facing now. Currently, they are often teaching workshop
like formats in research institutions.

\begin{itemize}
\tightlist
\item
  There are outreach issues. We emphasize that there are two dimensions
  to this: First it is important that we inform students that workshops
  exist, and then, the more important part, we also need to motivate
  people to invest the time for a workshop. \autocite{EuroCC2022}
\item
  Adaptation of material to the target audience has been identified as a
  time consuming task.
\item
  Organization and preparation is a challenge, since currently no
  standardized formats exist.
\item
  Expectation management of students. Existing knowledge of students is
  often diverse.
\item
  Language barriers. This can range from the use of technical jargon up
  to the disparities of you teaching in a foreign language.
\item
  Setting up a feedback loop that facilitates a reflection of the
  workshop for the teacher.
\item
  staying up-to-date with fast-moving RSE topics.
\end{itemize}

\hypertarget{what-mindset-makes-up-a-good-teacher}{%
\paragraph{What mindset makes up a good
teacher}\label{what-mindset-makes-up-a-good-teacher}}

Irrespective of where people come from they need to have the proper
mindset to properly foster aspiring RSEs.

\begin{itemize}
\tightlist
\item
  ``If you want to go fast, go alone; if you want to go far, go
  together?''
\item
  Not every ``good'' scientist wants to become a ``good'' software
  engineer, too!
\end{itemize}

\hypertarget{where-do-we-get-our-teachers-from}{%
\paragraph{Where do we get our teachers
from}\label{where-do-we-get-our-teachers-from}}

The community discussion brought about the need for a mixture of people,
thereby the education of aspiring RSEs will involve people from close
domain sciences or experienced RSEs and people that have respective
additional skills to teach RSE competencies to the new generation. In
that respect, this follows the carpentries model that offers
certifications but is still open to non-certified trainers. We highlight
and emphasize, that since a topic like RSE education, is constantly
evolving, trainers strongly require the opportunity to and the
recognition to educate themselves. Therefore our teachers will be
sourced from the workplace but there will also be certified RSE
Trainers. (FIXME: in classical university speak, these would be people
who have done their habilitation on that topic, right?)

We propose to create common Infrastructures that can be utilized in this
ongoing effort to professionalize the RSE education further, to easily
share education resources across the country. (FIXME: DETAIL ME
FURTHER!!!!)

\hypertarget{organization-of-teaching}{%
\subsubsection{Organization of
teaching}\label{organization-of-teaching}}

\hypertarget{certificates}{%
\paragraph{Certificates}\label{certificates}}

With the ever-growing demand for RSEs in science it could be helpful for
people to earn certificates for skills needed in certain RSE positions.
This would possibly make hiring easier and could incentivise researcher
to go through proper courses on these skills instead of learning on the
go. For certain skills it would also be good for finding jobs outside
academia, e.g.~in industry where certain practices are already
state-of-the-art. However, these certificates are only helpful if there
is a certain level of standardization, which would require a central
authority or collaboration between multiple stakeholders to decide on
contents and allow participating institutions to issue these
certificates. Additionally, it can be excluding capable applicants that
already use these skills but never got a certificate for it.

The possible types of certificates can of course differ. Badges are
increasingly popular, from personal badges rewarding contributions to a
specific community,
e.g.~\href{https://badges.fedoraproject.org/explore/badges}{Fedora
Badges} or
\href{https://github.com/Schweinepriester/github-profile-achievements}{Github
Achievements}, or project badges highlighting coding practices,
e.g.~Build and CI status or code coverage. RSEs are very likely subject
to life-long learning and personal, technical badges are one possibility
for older RSEs to showcase that they posess a certain technical skill.
Classical attendance sheets for courses are another option. To further
incentivise participation by students, some of these courses might also
award academic credit points like ECTS and benefit them on their way to
graduation.

Having certificates provides finally a clear understanding of which
tasks an RSE can perform and thereby helps defining the career path and
the job description. A big demand for specialized RSEs will certainly
come from the newly established RSE centers at research institutions
that require skilled people to fill their vacant positions. And using
the certificates, the demand can now be satisfied with people offering
this skill.

Some exemplary skills for which courses are already held are version
control tools like git, HPC topics like multithreading, MPI and GPU
computations, FAIR principles.

\hypertarget{a-possible-graduation-path-within-the-classical-university-structures}{%
\paragraph{A possible graduation path within the classical university
structures}\label{a-possible-graduation-path-within-the-classical-university-structures}}

We have put forward the idea that familiarity with research is a
prerequisite for an RSE in order to be able to work effectively in the
research space and in collaboration with researchers. In this particular
example, we consider a path into RSE via a traditional university route
involving Bachelors and Masters degree studies that include an RSE
element. However, we recognise that there are other routes into an RSE
career and these are increasing. For example, some RSEs come from an
industry background, others may come through apprenticeship or similar
programmes. In both cases, gaining knowledge of the research lifecycle
and understanding the ways that researchers work towards solutions to
research challenges is something that can be developed on-the-job
alongside training opportunities and the chance to work directly with
researchers. This leads naturally to the question, whether it is
possible to become an RSE without a home research domain. With software
being a core element of the research process in so many different
domains, it is not helpful for everybody working in RSE to have a
background in computing or software engineering. Indeed, we consider it
much more useful if new graduates looking to work in the RSE space come
from a wide range of different domains. Expertise, beyond software
development skills, in another research domain can be an important
element of an RSE team being able to support RSE projects in that
domain. Assuming for a moment, that people have done their masters
studies in a particular domain, e.g.~from the natural sciences, and that
we can assume that the lectures to that point contain a mixture of
domain specific content and RSE specific content (A good starting point
for an RSE in the natural sciences), then we come to the question of the
topic of the masters thesis. In order for young RSEs to get their
research experience we believe it is necessary that already in their
master's thesis young RSE students are given computational research
tasks that can be solved with the RSE specific skills of that domain.
This gives them a Master's degree of a \$\{DOMAIN\}-RSE that has learnt
in their lectures a research domain specific part and a software
engineering specific part, and enabled them to get a first dip into
actual science in their master's thesis. Of course, the next question
for their future is whether a master's degree enables them to really be
effective parts of a research group. While we accept this is something
of a generalisation, we argue that this is most likely not the case
since undertaking a PhD provides a much more extensive set of research
training and experience that can be vital for a researcher's future.
Research environments different internationally but in many cases there
are formal barriers in the research landscape that require a PhD
(e.g.~eligibility for funding). Hence a PhD is required to actively
participate in science and why we argue the regular RSE should do a PhD
that on one end combines knowledge of a research domain with software
engineering heavy task such that both pillars are suitably covered and
they were able to observe how research functions.

\hypertarget{specialised-masters-programs}{%
\paragraph{Specialised Master's
Programs}\label{specialised-masters-programs}}

On the other hand, when pursuing a PhD, scientists are increasingly
required to do RSE-type work as part of their research as data and
computation are becoming part of research tasks in a huge range of
fields. It is not uncommon for researchers to be faced with RSE topics
for the first time, because it has not been part of their academic
curricula. Many are faced with a steep learning curve that requires them
to invest a huge amount of time to catch up. Naturally, many would only
invest as much as necessary to get the job done regardless of whether
the solution is sustainable or not. Support from RSEs is one way to
resolve this challenge. Another would be to lay more effective
foundations for future RSE work at a much earlier stage in
undergraduate/postgraduate studies. We see scope for establishing
dedicated RSE Master's programmes which specialise in developing RSE
skills and practices. Some universities already offer dedicated master's
programs in some domains. Examples would be Computational Sciences in
Engineering (CSE) or Bioinformatics. Where appropriate similar programs
should also be established in other domains.

\hypertarget{required-next-steps}{%
\subsection{Required Next steps}\label{required-next-steps}}

\hypertarget{implementation-strategies}{%
\subsubsection{Implementation
Strategies}\label{implementation-strategies}}

\begin{itemize}
\tightlist
\item
  Ideally over time scientific software engineering becomes part of the
  curricula at universities.
\end{itemize}

\hypertarget{academic-considerations}{%
\paragraph{Academic Considerations}\label{academic-considerations}}

\begin{itemize}
\tightlist
\item
  Awareness of existing teaching programs
\item
  ``Branded'' Add-on courses
\item
  External Institutions provide resources
\item
  fully recognized in the academic system. Students get ECTS points.
\item
  Bachelor/Master specializations
\end{itemize}

\hypertarget{broader-considerations}{%
\paragraph{Broader Considerations}\label{broader-considerations}}

\begin{itemize}
\tightlist
\item
  Instilling more respect for people that want to educate themselves for
  digital competencies
\item
  Outreach to people that now have the feeling that they require this
  training.
\end{itemize}

\hypertarget{conclusion}{%
\subsection{Conclusion}\label{conclusion}}

We have indentified the RSE has an individual that contributes to
research teams with their knowledge about digital tools. Then we have
defined generic core competencies from the pillars of Software
Engineering, Research and Team processes. We fleshed them out with some
possible specializations of RSEs. Given the competencies and a
demand(FIXME: Do the calculation) in the research landscape for them we
moved on to define who the teachers are for this new field. We closed
with a discussion of possible structures and organization forms that
educate new generations of RSEs in more structured programs than what is
available today(FIXME: this is currently aspirational). Therefore this
closes the gap, that the research landscape requires RSEs, but there are
no structures where these persons are educated, by detailing the career
path that a young person might want to take to become an RSE.(FIXME:
also aspirational\ldots)

\hypertarget{appendix}{%
\subsection{Appendix}\label{appendix}}

\hypertarget{an-applied-example-curriculum}{%
\subsubsection{An applied example
curriculum}\label{an-applied-example-curriculum}}

\hypertarget{an-example-of-a-possible-career-path}{%
\subsubsection{An example of a possible career
path}\label{an-example-of-a-possible-career-path}}

\begin{itemize}
\tightlist
\item
  We can follow Kim, who has been the protagonist of the original deRSE
  Paper.
\end{itemize}

\printbibliography

\end{document}
